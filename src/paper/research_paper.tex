\documentclass[11pt, a4paper, leqno]{article}
\usepackage{a4wide}
\usepackage[T1]{fontenc}
\usepackage[utf8]{inputenc}
\usepackage{float, afterpage, rotating, graphicx}
\usepackage{epstopdf}
\usepackage{longtable, booktabs, tabularx}
\usepackage{fancyvrb, moreverb, relsize}
\usepackage{eurosym, calc}
% \usepackage{chngcntr}
\usepackage{amsmath, amssymb, amsfonts, amsthm, bm}
\usepackage{caption}
\usepackage{mdwlist}
\usepackage{xfrac}
\usepackage{setspace}
\usepackage{xcolor}
\usepackage{subcaption}
\usepackage{minibox}
\usepackage{graphicx, dblfloatfix}
% \usepackage{pdf14} % Enable for Manuscriptcentral -- can't handle pdf 1.5
% \usepackage{endfloat} % Enable to move tables / figures to the end. Useful for some submissions.



\usepackage{natbib}
\bibliographystyle{plainnat}



\usepackage[unicode=true]{hyperref}
\hypersetup{
    colorlinks=true,
    linkcolor=black,
    anchorcolor=black,
    citecolor=black,
    filecolor=black,
    menucolor=black,
    runcolor=black,
    urlcolor=black
}


\widowpenalty=10000
\clubpenalty=10000

\setlength{\parskip}{1ex}
\setlength{\parindent}{0ex}
\setstretch{1.5}


\begin{document}

\title{Particle Smoother Method in Estimating the Technology of Skill Formation  \thanks{Template of this study is by \citep{GaudeckerEconProjectTemplates}}}

\author{Elif Bodur \\ Maximilian Weiß}

\date{
\today
}

\maketitle


\begin{abstract}
	This study examines the performance of backward particle smoother analysis by applying the method to the model in Cunha et al. (2010). \nocite{cunha2010} Dataset is generated based on the simplified version of the model from the replication files of the study. Given that all parameters and 'true'  factors of the model are known, two analysis are conducted by which the prior sample is a random sample from the known distribution and the prior sample is the true factor values from the generated dataset. We compare the values of estimated factors to the values of true factors in each analysis. Having the true prior distribution does not contribute substantially to the estimation. In either analysis, article filtering method does not perform well in the model.... The findings has implications for the application of the particle filtering method to time varying model with small number of periods. 
\end{abstract}
\clearpage

\section{Introduction} % (fold)
\label{sec:introduction}

Filtering methods are widely used tools to estimate unobserved states of dynamic systems from noisy observed measurements. Gaussian approximations are preferred in many filtering problems. However, in case that the filtering distributions are multi-modal or that some of the state components are discrete, particle filtering based on sequential importance resampling is preferred over the Bayesian filtering methods\nocite{sarkka2013}. \par

In this project, we apply the backward-simulation particle smoother using the model-specification in Example 2 from the Cunha, Heckman, and Schennach replication files \footnote{The link to replication files is provided here \citep{chs_rep}}. Section II gives the model specification and data generation shortly. Section III describes the analysis. Section IV presents the results. Section V briefly discusses the results and concludes.

\section{Model Specification and Data} % (fold)
\label{sec:model spefication and data}

There exist three factors ($\alpha_1, \alpha_2, \alpha_3$) of which first and second ones are time varying whereas the third one is constant. At each period there are three separate measurements for each factor. It is also assumed that there are two independent control variables ($x_1, x_2$).  \par 
Transition equations defining the evolution of the factors are,
\begin{align}
& ln\alpha_{1, t+1}  = \displaystyle \frac{1}{\phi \lambda_1} ln \{ \gamma_{1, 1} e^{\phi \lambda_1 ln\alpha_{1, t}} + \gamma_{1, 2}  e^{\phi ln\alpha_{2, t}} \} + \eta_{1, t+1}, \\
& ln\alpha_{2, t+1} = \gamma_{2, 2} ln\alpha_{2, t} + \eta_{2, t+1}, \\
& \alpha_{3,t+1} = \alpha_{3,t}, \hspace{1cm} \text{for $t=1, 2, ..., 8$}
\end{align}
Measurement equations on factors have the following form,
\begin{align*}
y_{k, t} = \beta_{k, 1}x_1 + \beta_{k, 2}x_2 + Z_{k, t, 1}\alpha_{i, t} + \epsilon_{k, t} ,
\end{align*}
where k $\in \{1, 2, 3\}$ for factor 1 ($i=1$), k $\in \{4, 5, 6\}$ for factor 2 ($i=2$), and k $\in \{7, 8, 9\}$ for factor 3 ($i=3$). Measurements on factor 3 are time-invariant as factor 3 is.

Data generation follows from fixing the parameters of the model including all of the coefficients and variance of the errors as well as the exact prior distribution. Once we store all the measurements and factors for each period, we run the analysis.

Here comes the Max............
\section{Backward-Simulation Particle  Smoother}
\label{sec: bakward-simulation particle smoother}

\section{Results}
\label{sec: results}

$table1

\begin{figure}{h!}
	\centering
 		\caption{Factor 1 with randomly drawn prior}
\includegraphics[width=0.8\textwidth]{../../../bld/out/figures/rnd_prior_boxplot_fac1.png}
 	
\end{figure}

\begin{figure}{h!}
	\centering
 		\caption{Factor 2 with randomly drawn prior}
\includegraphics[width=0.8\textwidth]{../figures/rnd_prior_boxplot_fac2.png}
\end{figure}

\begin{figure}{h!}
	\centering
 		\caption{Factor 3 with randomly drawn prior}
\includegraphics[width=0.8\textwidth]{../../../bld/out/figures/rnd_prior_boxplot_fac3.png}
\end{figure}
\begin{figure}{h!}
	\centering
 		\caption{Factor 1 with true degenerate prior}
\includegraphics[width=0.8\textwidth]{../../../bld/out/figures/deg_prior_boxplot_fac1.png}

\end{figure}

\begin{figure}{h!}
	\centering
 		\caption{Factor 2 with true degenerate prior}
\includegraphics[width=0.8\textwidth]{../../../bld/out/figures/deg_prior_boxplot_fac2.png}
\end{figure}
\begin{figure}{h!}
	\centering
 		\caption{Factor 3 with true degenerate prior}
\includegraphics[width=0.8\textwidth]{../../../bld/out/figures/deg_prior_boxplot_fac3.png}
 
\end{figure}

\section{Conclusion}

\newpage











\clearpage


\bibliography{refs}



% \appendix

% The chngctr package is needed for the following lines.
% \counterwithin{table}{section}
% \counterwithin{figure}{section}

\end{document}