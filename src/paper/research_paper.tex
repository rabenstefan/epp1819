\documentclass[11pt, a4paper, leqno]{article}
\usepackage{a4wide}
\usepackage[T1]{fontenc}
\usepackage[utf8]{inputenc}
\usepackage{float, afterpage, rotating, graphicx}
\usepackage{epstopdf}
\usepackage{longtable, booktabs, tabularx}
\usepackage{fancyvrb, moreverb, relsize}
\usepackage{eurosym, calc}
% \usepackage{chngcntr}
\usepackage{amsmath, amssymb, amsfonts, amsthm, bm}
\usepackage{caption}
\usepackage{mdwlist}
\usepackage{xfrac}
\usepackage{setspace}
\usepackage{xcolor}
\usepackage{subcaption}
\usepackage{minibox}
% \usepackage{pdf14} % Enable for Manuscriptcentral -- can't handle pdf 1.5
% \usepackage{endfloat} % Enable to move tables / figures to the end. Useful for some submissions.



\usepackage{natbib}
\bibliographystyle{plainnat}



\usepackage[unicode=true]{hyperref}
\hypersetup{
    colorlinks=true,
    linkcolor=black,
    anchorcolor=black,
    citecolor=black,
    filecolor=black,
    menucolor=black,
    runcolor=black,
    urlcolor=black
}


\widowpenalty=10000
\clubpenalty=10000

\setlength{\parskip}{1ex}
\setlength{\parindent}{0ex}
\setstretch{1.5}


\begin{document}

\title{Particle Smoother Method in Estimating the Technology of Skill Formation  \thanks{Template of this study is by \citep{GaudeckerEconProjectTemplates}}}

\author{Elif Bodur \\ Maximilian Weiß}

\date{
\today
}

\maketitle


\begin{abstract}
	This study examines the performance of backward particle smoother analysis by applying the method to the model in Cunha et al. (2010). \nocite{cunha2010} Dataset is generated based on the simplified version of the model from the replication files of the study. Given that all parameters and 'true'  factors of the model are known, two analysis are conducted by which the prior sample is a random sample from the known distribution and the prior sample the true factors from the generated dataset. We compare the estimated factors to the true factors in each analysis. Having the true prior distribution does not contribute substantially to the estimation. In either analysis, article filtering method does not perform well in the model.... The findings has implications for the application of the particle filtering method by presenting an evidence that it does not perform in time varying model, which supports the claim by Cunha et al. (2010). \nocite{cunha2010}
\end{abstract}
\clearpage

\section{Introduction} % (fold)
\label{sec:introduction}



 \citet{GaudeckerEconProjectTemplates}







\clearpage


\bibliography{refs}



% \appendix

% The chngctr package is needed for the following lines.
% \counterwithin{table}{section}
% \counterwithin{figure}{section}

\end{document}
